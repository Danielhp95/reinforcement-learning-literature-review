\documentclass{../main.tex}{}
\begin{document}
% Inspiration: https://raw.githubusercontent.com/zkid18/RL-Study-Roadmap/master/header_image.png
% Looks like a decent read: https://towardsdatascience.com/introduction-to-various-reinforcement-learning-algorithms-i-q-learning-sarsa-dqn-ddpg-72a5e0cb6287


Every RL algorithm attempts to learn an optimal policy $\pi^*$ for a given environment $\xi$. So far, there is not a single algorithm which is used in every single environment to find an optimal policy. The choice of algorithm depends on many factors, such as the nature of the environment, the availability of the underlaying mechanics of the environment, access to already existing policies and more practical constraints such as the amount of computational power available. More importantly, RL algorithms are not, against common belief, black boxes Most RL algorithms can be divided into the following categories, note that not all of them are mutually exclusive:

\subsection{On-policy and off-policy algorithms}

If any RL algorithm can be regarded as a learning function mapping state-action-reward sequences, also known as paths or trajectories, to a policies~\citep{Laurent2011}. Esentially all that RL algorithms do is applying a learning function over paths sampled from the environment and a policy. The key and only difference between on-policy and off-policy algorithms is the following:
\begin{itemize}
    \item On-policy algorithms use the policy that they are learning about to sample actions in the environment. A policy $\pi$ is both being improved overtime\footnote{The notion of improvement overtime is expressed as a monotonic increase in the expected reward of an episode $\mathbb{E}_{a \sim \pi_0}[\sum_{t=0}^{\infty}r_t] < \mathbb{E}_{a \sim \pi_1}[\sum_{t=0}^{\infty}r_t]$} and also used to sample actions $a \sim \pi(s)$ inside of the environment.
    \item Off-policy algorithms use a behavioural policy $\mu$ to sample actions $a \sim \mu(s)$ and paths inside of the environment, and use this information to improve a target policy $\pi$. The learning that takes place in off-policy algorithms can be regarded as learning from somebody else's experience, whilst on policy algorithms focus on learning from an agent's own experience.
\end{itemize}

On-policy algorithms dedicate all computational power on learning and using a single policy $\pi$. These methods can therefore focus on spending the computational resources on applying a learning function for the sole benefit of improving this policy. With off-policy algorithms it is possible to dedicate computational time to modifying both the behavioural policy $\mu$ and the policy $\pi$ being learnt. The motivation behind this being that the paths sampled from the environment using $\pi$ won't necessarily yield the best paths to learn from. However, by spending some of the computational resources to using, or even changing, the behavioural policy $\mu$, it is possible to generate more ``imformative'' trajectories with which we can improve $\pi$ through a learning function.

Can be used to learn multiple tasks in parallel.~\cite{Sutton2010} use sensorimotor interaction with an environment to learn a multitude of pseudoreward funcitons.~\cite{Jaderberg2016} (UNREAL) takes this idea further by using an off-policy malgorithms to learn auxiliary extra tasks. Most notably, these tasks include predicting immediate rewards\footnote{This is different from value function estimation because the value that the off-policy algorithm is trying to predict is expected immediate reward, instead of expected future cummulative reward.}, pixel control\footnote{Given a matrix of pixels as input, the authors define pixel control as a separate policy that tries to maximally change the pixels in the following state. The reasoning behind this approach is that big changes in pixel values may correspond to important events inside of the environment.}.

(TODO rephrase)
A method to allow algorithms to perform off-policy updates to their policies is to introduce the notion of an \textit{experience replay}~\citep{Lin1993}, which was made famous after the success of~\cite{Mnih2013}. An experience replay is a list of experiences, where each experience is a 5 element tuple $<s_t, a_t, r_t, s_{t+1}, a_{t+1}>$. As an agent acts in an environment, in the same fashion as in the reinfocement learning loop presented in (reference), the experience replay is filled. At the time of updating the policy, the agent does not choose to update its policy using the last action function that was taken, as it is the case with Q-learning. Instead, the agent samples an experience (or batch of experiences) from the experience replay. Because these sampled experiences may have been generated using a previous policy, experience replay allows for policy updates to happen in an off-policy fashion. The experience replay buffer has been the focus on future research such as \citep{Schaul2015, Hessel2017} where the authors use a \textit{prioritized} experience replay. The difference is that experiences are not sampled uniformly from the replay buffer. Instead, experiences are weighted according to (READ paper).~\cite{Andrychowicz2017} expands on the idea by introducing the Hindsight Experience Replay (HER) where an experience replay is used to learn a task from failures by treating certain state transitions as goals.

(DDPG uses a stochastic exploratory function whilst learning a deterministic function).
UNREAL~\citep{Jaderberg2016} algorithm optimizes a loss function with respect to the joint parameters $\theta$, that combines the A3C loss $\mathcal{L}_{A3C}$ together with an auxiliary pixel control loss $\mathcal{L}_{PC}$, auxiliary reward prediction loss $\mathcal{L}_{RP}$ and replayed value loss $\mathcal{L}_{VR}$, with loss weightings $\lambda_{PC}$, $\lambda_{RP}$, $\lambda_{VR}$.

Off-PAC, The first off-policy policy gradient method introduced by~\cite{Degris2012} used importance sampling techniques to weigh the actor gradient update against the behavioural policy being used. They also used eligibility traces for a critic with linear function approximator, similar to a TD($\lambda$). Reply buffer, introduced in~\cite{Lin1993}, has seen a lot of use recently~\citep{Mnih2013, Mnih2016}. % chktex 36

% About DQN: DQN is able to learn value functions using such function approximators in a stable and robust way due to two innovations: 1. the network is trained off-policy with samples from a replay buffer to minimize correlations between samples; 2. the network is trained with a target Q network to give consistent targets during temporal difference backups.

% Notes:
% good post on DDPG: http://pemami4911.github.io/blog/2016/08/21/ddpg-rl.html
%(REPHRASE) One challenge when using neural networks for reinforcement learning is that most optimization al- gorithms assume that the samples are independently and identically distributed. Obviously, when the samples are generated from exploring sequentially in an environment this assumption no longer holds. This can be solved by using a mechanism that breaks down the dependency of samples, such as an experience replay.

\textbf{Famous on-policy algorithms:} Sarsa~\citep{Sutton1998}, $Q(\sigma)$~\citep{Deasis2017}, Monte Carlo Tree search (MCTS), REINFORCE~\citep{Williams1992}, Asynchronous Advantage estimation Actor Critic (A3C).

\textbf{Famous off-policy algorithms:} Q-learning, Deep Q-Network (DQN)~\citep{Mnih2013}, Deterministic Policy Gradient (DPG)~\citep{Silver2014}. Deep Deterministic Policy Gradient (DDPG)~\citep{Lillicrap2015}, Importance Weighted Actor-Learner Architecture (IMPALA)~\citep{Espeholt2018}.


% Reinforcement Learning algorithms which are characterized as off-policy generally employ a separate behavior policy that is independent of the policy being improved upon; the behavior policy is used to simulate trajectories. A key benefit of this separation is that the behavior policy can operate by sampling all actions, whereas the estimation policy can be deterministic (e.g., greedy) [1]. Q-learning is an off-policy algorithm, since it updates the Q values without making any assumptions about the actual policy being followed. Rather, the Q-learning algorithm simply states that the Q-value corresponding to state s(t) and action a(t) is updated using the Q-value of the next state s(t+1) and the action a(t+1) that maximizes the Q-value at state s(t+1).
% 
% On-policy algorithms directly use the policy that is being estimated to sample trajectories during training.

\subsection{Value based}
Value based, also known as critic only methods, rely on deriving a policy $\pi$ from a state value function $V(s)$ or a state action value function $Q(s,a)$. These include most if not all of the traditional RL algorithms. There are various methods for extracting a policy from a value function. The simplest form of deriving a policy from a value function is to create a policy that acts greedily w.r.t a value function:

\begin{equation}
    \label{equation:policy-extraction}
\begin{aligned}
    \forall s \in \mathcal{S}: \quad \pi(s) &= \argmax_{a} \sum_{s' \in Succ(s)} P(s' \mid s, a) V(s') \quad & (\text{Deriving policy from } V(s)) \\
    \forall s \in \mathcal{S}: \quad \pi(s) &= \argmax_{a} Q_{\pi}(s,a) \quad & (\text{Deriving policy from } Q(s,a))
\end{aligned}
\end{equation}

Some algorithms such as Q-learning, covered in Section~\ref{section:q-learning}, and SARSA bootstrap a state action value function $Q(s,a)$ towards the value functions of an optimal policy, $Q_{\pi^*}(s,a)$. Temporal Difference (TD) algorithms bootstrap a state value function $V(s)$ towards the value function of an optimal policy, $V_{\pi^*}(s)$. These algorithms have been proved to converge to the optimal value functions, so an optimal policy $\pi^*$ can be derived once the bootstraping process converges.

Other methods, like Value iteration or Policy iteration go through an iterative loop of \textit{policy evaluation} and \textit{policy improvement}. The policy evaluation step computes a value function $V_{\pi}(s)$ or $Q_{\pi}(s,a)$ for a given a policy $\pi$, which is randomized on initialization. The policy improvement step extracts a new policy $\pi'$ from the pre-computed value functions $V_{\pi}(s)$ or $Q_{\pi}(s,a)$ using equation~\ref{equation:policy-extraction}. The next iteration of the loop is computed using $\pi'$. This two step process is proved to converge to both the optimal value function and optimal policy.

\textbf{Famous value based algorithms:} Value iteration, Policy iteration, SARSA, TD(0), TD(1), TD($\lambda$)~\citep{Sutton1998}, Q-learning, DQN and it's many variants: Dueling DQN, Distributional DQN, Prioritized DQN and Double DQN\@. These can be found in~\citep{Hessel2017}

\subsection{Policy based}
(they don't use value functions at all. Parameterize a policy, iteratively change parameters in direction of improvement)
These will be heavily covered in Section~\ref{section:policy-gradient-methods}. These algorithms tend to represent a policy through a parameter vector $\theta
\in \mathbb{R}^D$. The goal then becomes to improve the choice of parameters
$\theta_i \in \theta$ to improve the expected sum of rewards
$\mathbb{E}[\sum_{T} r(t)]$. Policy based algorithms are the main focus on Section~\ref{section:policy-gradient-methods}.

\textbf{Famous policy based algorithms:} vanilla policy gradient, REINFORCE\citep{Williams1992} and the REINFORCE family of algorithms\@.

\subsection{Actor-critic}

(Shamelessly copy from the Konda2000 paper.
Actor-critic methods combine the strong points of both policy based and value based algorithms, and overcome some of their individual weaknesses~\citep{Konda2000}. The critic assumes the role of learning a value function, which is then used as part of the update for the actor's policy. The individual critic is analogous to value based algorithms, and the actor to policy based methods.

~\cite{Konda2000} make the key observation that in actor critic methods, the actor parameterization $\theta$ and the critic parameterization $\phi$ should \textit{not} be independent. The choice of critic parameters should be directly prescribed by the choice of the actor parameters. That is why all real world applications that implement a policy gradient algorithm using an estimated baseline use a unique parameterized model (e.g. a neural network) to represent both the policy and the baseline, to share parameters between actor and critic.

\textbf{Famous actor critic algorithms:} A3C~\citep{Mnih2016}, PPO~\citep{Schulman2017}, TRPO~\citep{Schulman2015}, ACKTR~\citep{Wu2017}.

\subsection{Model based and model free approaches}
(Just small explanation, then scrap most of it)
In the reinforcement learning literature the \textit{model} or the dynamics of an environment is considered to be the transition function $\mathcal{P}(s_{t+1} | s_t, a_t)$ and reward function $\mathcal{R}(s_t, a_t, s_{t+1})$. Model free algorithms aim to approximate an optimal policy $\pi^*$ without explicitly using either $\mathcal{P}$ or $\mathcal{R}$ in their calculations.

A further categorization of algorithms is the notion of \textit{model free} and \textit{model based} algorithms. Consider a \textit{model} of an environment to be the transition function $\mathcal{P}$ and reward function $\mathcal{R}$. Model free algorithms aim to approximate an optimal policy without them. Model based algorithms are either given a prior model that they can use for planning \citep{browne2012survey, Soemers2014}, or they learn a representation via their own interaction with the environment \citep{Sutton1991, Guzdial2017}. 

Curiosity! Learning a model of the environment to generate intrinsic reward (Reward shaping). \citep{Pathak2017}.

Model based algorithms are either given a prior model that they can use for planning \citep{browne2012survey, Soemers2014}, or they learn a representation via their own interaction with the environment \citep{Sutton1991, Guzdial2017, Deisenroth2011}. Note that an advantage of learning your own model is that you can choose a representation of the environment that is relevant to the agent's actions, which can have the advantage of modelling uninteresting (but perhaps overly complicated) environment behaviour \citep{Pathak2017}. Another advantage of having a model is that it allows for forward planning, which is the main method of learning for search-based artificial intelligence (throw mcts papers here).



\end{document}
